\section{L'Arno mormorava}
\subtitle{Sulla melodia de “La canzone del Piave“}
\begin{canzone}
L'Arno mormorava calmo e placido di fronte
Ai primi allievi in marcia sopra il ponte.
L'esercito avanzava, trasportava i propri scudi
Mostrando al vento i fieri petti ignudi.
Carrelli ovunque pien di gavettoni
Andavano a lavar quei gran puzzoni!
S'udiva intanto dalle vie d'intorno
Il forte canto ed il suonar del corno
Era il gran grido delle invitte schiere
E l'Arno mormorò:
Non passa l'ingegnere!

Ma dopo la battaglia si parlò di codardia
Ch'era il nemico presto corso via.
Saliva la marmaglia sulle scale in Cavalieri
Dava una foto in pasto a dei ciarlieri
Giornali scrive certo non la storia
Il santannino che grida vittoria
Veniva intanto in mente a ogni buffone
Il tempo dell'esame d'ammissione
Credeva d'esser genio di sapere
Ma il test comandò:
Segato è l'ingegnere!

E ritornò una sera in San Francesco la tenzone
Si miser tutti quanti in formazione.
S'ergeva tra la schiera maestoso un gran fortino
Sottrasse ogni speranza al santannino!
Carrelli ultrà e scudi ancor più grossi
E il fuoco e i calici scottanti e rossi!
Cercò la feccia allor di rimediare
Le scale in carovana di occupare
Ma i normalisti eran lì a tenere.
La piazza comandò:
Indietro va' ingegnere!

E indietreggiò sconfitto fino in Santa Caterina
Si chiuse nella sua volgar latrina.
Placatosi il conflitto tra le schiere il costruttore
Apparve assieme al nostro Gran Priore!
Sancì allora questa sacra vista
La nobile vittoria normalista!
Sicura casa e libere le scale
E tacque l'Arno e si zittì il giornale.
Sul patrio suolo vinti gl'ingegneri
La scienza non trovò
Spazzini e parrucchieri!
\end{canzone}
